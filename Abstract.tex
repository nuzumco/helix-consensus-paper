We present \emph{\nameNS}, a Byzantine fault tolerant and scalable consensus algorithm for fair ordering of transactions among nodes in a distributed network. In \nameNS, one among the network nodes proposes a potential set of successive transactions (block). The known PBFT protocol is then run within a bounded-size committee in order to achieve agreement and commit the block to the blockchain indefinitely. 
%Scalability is facilitated under limited committees as the network enlarges.
In \nameNS, transactions are encrypted via a threshold encryption scheme in order to hide information from the ordering nodes, limiting censorship and front-running. The encryption is further used to realize a verifiable source of randomness, which in turn is used to elect the committees in an unpredictable way, as well as to introduce a correlated sampling scheme of transactions included in a proposed block. The correlated sampling scheme restricts nodes from promoting their own transactions over those of others.
%Nodes are elected to participate in committees in proportion to their relative reputation, measuring nodes' obedience to the protocol instructions. This guarantees the fair composition of committees.   
Nodes are elected to participate in committees in proportion to their relative reputation. Reputation, attributed to each node, serves as a measure of obedience to the protocol's instructions. Committees are thus chosen in a way that is beneficial to the protocol.